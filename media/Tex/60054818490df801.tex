\documentclass[preview]{standalone}

\usepackage[english]{babel}
\usepackage[utf8]{inputenc}
\usepackage[T1]{fontenc}
\usepackage{amsmath}
\usepackage{amssymb}
\usepackage{dsfont}
\usepackage{setspace}
\usepackage{tipa}
\usepackage{relsize}
\usepackage{textcomp}
\usepackage{mathrsfs}
\usepackage{calligra}
\usepackage{wasysym}
\usepackage{ragged2e}
\usepackage{xcolor}
\usepackage{microtype}
\DisableLigatures{encoding = *, family = * }
\linespread{1}

\begin{document}

\begin{center}
Geometrically, we have a linear hyperplane spanned by $\vec{a}_1,\vec{a}_3$\\and this hyperplane divides the whole space into two parts.\\We are then finding $\vec{c}$ orthogonal to the hyperplane\\such that $\vec{c}$ lies on the same side as $\vec{a}_2$.
\end{center}

\end{document}
